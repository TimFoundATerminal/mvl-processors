\documentclass[12pt]{article}
\usepackage[utf8]{inputenc}
\usepackage[margin=1in]{geometry}
\usepackage{graphicx}
\usepackage{amssymb}
\usepackage{hyperref}
\usepackage{xcolor}
\usepackage{listings}
\usepackage{color}
\usepackage{amsfonts}
\usepackage{amsmath}
\usepackage{csquotes}
\usepackage{epigraph}

% Title formatting
\title{\textbf{A Comparative Analysis of the Performance of Emulated Ternary vs Binary Logic Systems} \\ [1ex] \large Literature, Technology and Data Survey}

\author{
    Thomas Timmons\\
    Bachelor Hons. Computer Science\\
    \href{mailto:tlt33@bath.ac.uk}{tlt33@bath.ac.uk}
}

\begin{document}

\maketitle

\section{Introduction}

Binary Logic has been the defacto standard for digital systems since the inception of the transistor. However, due to the 
imminent end of Moore's Law, researchers have been looking for alternative methods to increase the performance of digital
systems. This paper will attempt to realise the theoretical performance gains of ternary logic within an emulator. We can 
then measure and project the performance on the assumption it should be possible to implement these circuits in hardware 
and discuss possible implications.

\section{Literature and Technology Survey}

James' Feedback\\
\\
Technology Survey:\\
Ways to implement emulators in software.\\
Review Tools and explain why implementation by yourself is 
more comparable.\\

This report will provide a comprehensive review of the literature and technology surrounding the following topics:
\begin{itemize}
    \item Ternary Algebra
    \item Ternary Logic Gates
    \item Previous Ternary Implementations
    \item Emulation of Circuits
\end{itemize}
Each of these areas will be explored in detail while assessing the credibility and motivation for the work. The report will 
aim to build the foundation on which the project will be based.\\

\subsection{Ternary Algebra}

The cost or complexity of number systems can be evaluated through representation efficiency,
computational overhead, and implementation requirements.\\
\\
Representation efficiency refers to how digits or symbol are required represent any given range of values $N$, with the base of the number system represented 
by $R$ and the necessary number of digits required is $d$ (rounded upwards to the nearest integer):

\begin{equation}
  N = R^{d}
\end{equation}

This directly impacts the memory usage and storage costs in a computing system.\\
\\
The computational overhead involves the complexity of performing basic arithmetic operations. Larger bases require the managing of more states, requiring the 
processing of a larger set of possible combinations. The number of possible combinations is given by:

\begin{equation}
  C = R^{i}
\end{equation}

Where $i$ is the number of inputs for the operation and $R$ is the base of the number system as defined before. For example, the number of possible combinations 
for addition is $C = R^{2}$ which in binary is $C = 2^{2} = 4$ creating the set $\{0+0, 0+1, 1+0, 1+1\}$. 
In ternary the set would be $C = 3^{2} = \{0+0, 0+1, 0+2, 1+0, 1+1, 1+2, 2+0, 2+1, 2+2\} = 9$ which requires implementing more logic circuits to handle the 
additional combinations.\\
\\
Implementation requirements consider the the practical aspects of building a system that works with a given base. This includes 
error detection and correction using approaches such as Hamming Code's in binary logic % \citep{hamming1950error}.
As this paper aims to explore the performance of ternary logic systems in comparison to binary systems within an emulator, we will not be considering 
the practical challenges of implementing ternary logic in hardware.\\
\\

TODO: understand how this logic below impacts the performance of the ternary logic system.

The cost or complexity of the system $C$ is directly proportional to the capacity of the digits $(R \cdot d)$.
Therefore for some constants $k$ and $\log N$
% \citep{hurst1984mvl}:

\begin{equation}
  C = k(R \cdot d) = k \left( R \cdot \frac{\log N}{\log R} \right) = k \log N \left(\frac{R}{\log R}\right)
\end{equation}

Minimising this cost $C$ can be done by differentiating with respect to $R$ and setting the result to $0$:

\begin{equation}
  \frac{\partial C}{\partial R} = 
  k\log N\cdot\frac{d}{dR}\left[\frac{R}{\log R}\right] = 
  k\log N\cdot \frac{\log R -1}{(\log R)^2} = 0
\end{equation}

After removing constants $\log R = 1$ and solving for $R = e = 2.718$. Since as stated above, the radix $R$ must be an 
integer then we are left with $R = 3$. Therefore, the ternary logic system is the most efficient.\\
% \citep{jaber2020mvl}.
\\
In binary logic set of possible values is denoted as $\mathbb{B} = \{0, 1\}$. Ternary numeral systems can either unbalanced 
ternary values $\mathbb{T} = \{0, 1, 2\}$ or balanced ternary values $\mathbb{T} = \{-1, 0, 1\}$ with 
a variety of glyphs used to represent the values where $\overline{1}$ or $T$ is sometimes used to represent $-1$.\\
\\
Balanced ternary has the elegant property that of representing negative numbers intrinsically without the need for a 
separate signed bit. This leads to more symmetric carry rules due to a more even distribution around zero as shown in 
Figure \ref{fig:ternary_ops}.\\

\begin{figure}[h]
    \centering
    \begin{minipage}{0.22\textwidth}
        \centering   
        \begin{tabular}{|c||c|c|c|}
            \hline
            $+$ & $T$ & $0$ & $1$ \\
            \hline\hline
            $T$ & $T1$ & $T$ & $0$ \\
            \hline
            $0$ & $T$ & $0$ & $1$ \\
            \hline
            $1$ & $0$ & $1$ & $1T$ \\
            \hline
        \end{tabular}
    \end{minipage}
    \begin{minipage}{0.22\textwidth}
        \centering   
        \begin{tabular}{|c||c|c|c|}
            \hline
            $-$ & $T$ & $0$ & $1$ \\
            \hline\hline
            $T$ & $0$ & $T$ & \textbf{$T1$} \\
            \hline
            $0$ & $1$ & $0$ & $T$ \\
            \hline
            $1$ & \textbf{$1T$} & $1$ & $0$ \\
            \hline
        \end{tabular}
    \end{minipage}
    \begin{minipage}{0.22\textwidth}
        \centering   
        \begin{tabular}{|c||c|c|c|}
            \hline
            $\times$ & $T$ & $0$ & $1$ \\
            \hline\hline
            $T$ & $1$ & $0$ & $T$ \\
            \hline
            $0$ & $0$ & $0$ & $0$ \\
            \hline
            $1$ & $T$ & $0$ & $1$ \\
            \hline
        \end{tabular}
    \end{minipage}
    \begin{minipage}{0.22\textwidth}
        \centering   
        \begin{tabular}{|c||c|c|c|}
            \hline
            $\div$ & $T$ & $1$ \\
            \hline\hline
            $T$ & $1$ & $T$ \\
            \hline
            $0$ & $0$ & $0$ \\
            \hline
            $1$ & $T$ & $1$ \\
            \hline
        \end{tabular}
    \end{minipage}
    \caption{Balanced Ternary Single-Trit Addition, Subtraction, Multiplication and Division Tables}
    \label{fig:ternary_ops}
\end{figure}
For subtraction and division which are not commutative, the first operand is given to the left of the table and the second at the top.\\
\\
Balanced ternary has a range of advantages when it comes to computation over binary logic such as the reduction in the carry rate for 
addition/subtraction cuts down the carry rate in multi-digit multiplication as demonstrated in Figure \ref{fig:ternary_ops} with 
the values being $2/9$ and $1/4$ respectively. Also notice that it is possible to perform subtraction by negating the second operand and 
adding the two values together which will useful for the implementation of the ternary logic gates.\\
\\
Therefore, for this paper and emulator we will be using balanced ternary logic and any mention of `ternary logic` is referring to 
`balanced ternary logic`.\\
\\
To calculate the value of a ternary number $T$ with a radix point in decimal we can use the following formula:\\
TODO: this is not right and needs to be fixed.\\
\begin{equation}
    v = (a_{n}a_{n-1} \dotsb a_{0}.c_{1}c_{2}c_{3} \dotsb)_3 = \sum_{i=0}^{n} a_{i} 3^{i} + \sum_{i=1}^{\infty} c_{i} 3^{-i}
\end{equation}
Where $n$ and $m$ are the number of digits to the left and right of the radix point respectively and $v$ is the value in decimal given 
a vector of values containing a radix point.\\

To directly compare between the binary and ternary systems we need to consider how we can encode binary values into ternary values and 
vice versa. Using continued fractions, the best approximations to rationals are given by convergents to the continued fraction, 
alternating between overshooting and undershooting the target value
% \citep{davenport2008arithmetic}.\\
\\
Since we are looking for $2^{a} \approx 3^{b}$ where $a,b \in \mathbb{N}$. We can take the logarithm of both sides to get the following:
\begin{equation}
    \frac{a}{b} \approx \frac{\log 3}{\log 2}
\end{equation}
Where $2^{a} < 3^{b}$ to ensure that the binary value can be encoded into a ternary value.\\
\\
This continued fraction expansion can be calculated by computing the convergents which are as follows:
\begin{equation}
    \frac{\log 3}{\log 2} = a_0 + \frac{1}{a_1 + \frac{1}{a_2 + \frac{1}{a_3 + \dotsb}}}
\end{equation}
\begin{equation}
    {a_0; a_1, a_2, a_3, a_4 \dotsb} = {1; 1, 1, 2, 2, \dotsb}
\end{equation}

This gives us:
\begin{align*}
    \frac{2}{1} &= 1 + \frac{1}{1} \Rightarrow 2^{2} \approx 3^{1} \text {  } (\text{but ternary worse than binary: } 2^{2} > 3^{1})\\
    \frac{3}{2} &= 1 + \frac{1}{1 + \frac{1}{1}} \Rightarrow 2^{2} \approx 3^{1} \\
    \frac{8}{5} &= 1 + \frac{1}{1 + \frac{1}{1 + \frac{1}{1}}} \Rightarrow 2^{2} \approx 3^{1} \text {  } (\text{but ternary worse than binary: } 2^{2} > 3^{1})\\
    \frac{19}{12} &= 1 + \frac{1}{1 + \frac{1}{1 + \frac{1}{1 + \frac{1}{2}}}} \Rightarrow 2^{2} \approx 3^{1}
\end{align*}

$\frac{19}{12}$ provides a good approximation with an efficiency of 98.65\%. However using a 19 bit word is not practical due to it being 
too large.\\
\\
$\frac{11}{7}$ also provides a good approximation with an efficiency of 93.64\%. This is created by summing the 2nd and 3rd convergents
$\frac{3+8}{2+5}$ which provides a good balance between efficiency and word size.\\
\\
This is the same radio that Micron uses to encode 11 bits of binary data into 7 trits in their GDDR7 memory bus using PAM3.
% \citep{micron2024gddr7}.\\
For our emulator we will use the same ratio in an attempt to simulate the performance potential of running this bus directly into a ternary ALU.\\

\subsection{Ternary Logic Gates}

Gates can be represented in matrix form where the following Binary AND truth table has the has the equivalent matrix representation:\\
\begin{figure}[h]
    \centering
    \begin{minipage}{0.2\textwidth}
        \centering   
        \begin{tabular}{|c||c|c|}
            \hline
            $A \times B$ & $0$ & $1$ \\
            \hline\hline
            $0$ & $0$ & $0$ \\
            \hline
            $1$ & $0$ & $1$ \\
            \hline
        \end{tabular}
    \end{minipage}
    \begin{minipage}{0.2\textwidth}
        \centering   
        $\begin{bmatrix}
            0 & 0 \\
            0 & 1 \\
        \end{bmatrix}$
    \end{minipage}
    \caption{Binary AND Gate Truth Table and Matrix Representation}
    \label{fig:matrix_notation}
\end{figure}

\subsubsection{Single-Input Gates}

Single-input gates can be represented as a vector of length $2$ for binary logic and length $3$ for ternary logic. Combination theory 
shows that the number of possible combinations for a single input gate is $n^{n}$ where $n$ is the number of inputs. Therefore, the 
number of possible combinations for a single input gate is $2^{2} = 4$ and $3^3 = 9$ for binary and ternary logic gates. Most of these 
are redundant so only the useful gates have been listed in Figure \ref{fig:single_value_gates}.\\

\begin{figure}[h]
    \centering
    \begin{tabular}{c|c|c|c}
        Matrix & Description & Name & Symbol \\
        \hline
        $\begin{bmatrix}1 \\ 0 \\\end{bmatrix}$ & Identity, Buffer, Pass & Buffer & $A$ \\
        $\begin{bmatrix}1 \\ 0 \\\end{bmatrix}$ & Inverter & NOT & $\overline{A}$ \\
    \end{tabular}
    \caption{Useful Single Value Binary Gates}
    \label{fig:single_value_bin_gates}
\end{figure}

\begin{figure}[h]
    \centering
    \begin{tabular}{c|c|c|c}
        Matrix & Description & Name & Symbol \\
        \hline
        $\begin{bmatrix}T \\ 0 \\ 1 \\\end{bmatrix}$ & Identity, Buffer, Pass & BUF & $A$ \\
        $\begin{bmatrix}1 \\ 0 \\ T \\\end{bmatrix}$ & Inverter & NOT & $\overline{A}$ \\
        $\begin{bmatrix}1 \\ 1 \\ T \\\end{bmatrix}$ & Positively Biased Inverter & PNOT & $\hat{A}$ \\
        $\begin{bmatrix}1 \\ T \\ T \\\end{bmatrix}$ & Negatively Biased Inverter & NNOT & $\check{A}$ \\
        % $\begin{bmatrix}1 \\ 0 \\ 1 \\\end{bmatrix}$ & Absolute & ABS & $|A|$ \\
        % $\begin{bmatrix}T \\ 0 \\ 1 \\\end{bmatrix}$ & Increment & INC & $A^{+}$ \\
        % $\begin{bmatrix}T \\ 0 \\ 1 \\\end{bmatrix}$ & Decrement & DEC & $A^{-}$ \\
        % $\begin{bmatrix}T \\ 0 \\ 1 \\\end{bmatrix}$ & Is Positive & ISP & $A^{=+}$ \\
        % $\begin{bmatrix}T \\ 0 \\ 1 \\\end{bmatrix}$ & Is Zero & ISZ & $A^{=0}$ \\
        % $\begin{bmatrix}T \\ 0 \\ 1 \\\end{bmatrix}$ & Is Negative & ISN & $A^{=-}$ \\
    \end{tabular}
    \caption{Useful Single Value Binary and Ternary Gates}
    \label{fig:single_value_tern_gates}
\end{figure}

\subsubsection{Two-Input Gates}

\begin{figure}[h]
    \centering
    \begin{tabular}{c|c|c|c}
        Matrix & Description & Name & Symbol \\
        \hline
        $\begin{bmatrix}0 & 0 \\ 0 & 1 \\\end{bmatrix}$ & And & AND & $A \times B$ \\
        $\begin{bmatrix}0 & 1 \\ 1 & 1 \\\end{bmatrix}$ & Or & OR & $A + B$ \\
        $\begin{bmatrix}0 & 1 \\ 1 & 0 \\\end{bmatrix}$ & Exclusive Or & XOR & $A \oplus B$ \\
    \end{tabular}
    \caption{Useful Two-Input Binary Gates}
    \label{fig:two_value_bin_gates}
\end{figure}

\begin{figure}[h]
    \centering
    \begin{tabular}{c|c|c|c}
        Matrix & Description & Name & Symbol \\
        \hline
        $\begin{bmatrix}T & T & T \\ T & 0 & 0 \\ T & 0 & 1 \\\end{bmatrix}$ & And/Minimum & AND & $A \times B$ \\
        $\begin{bmatrix}T & 0 & 1 \\ 0 & 0 & 1 \\ 1 & 1 & 1 \\\end{bmatrix}$ & Or/Maximum & OR & $A + B$ \\
        $\begin{bmatrix}T & 0 & 0 \\ 0 & 0 & 0 \\ 0 & 0 & 1 \\\end{bmatrix}$ & Consensus & CONS & $A \boxtimes B$ \\
        $\begin{bmatrix}T & T & 0 \\ T & 0 & 1 \\ 0 & 1 & 1 \\\end{bmatrix}$ & Any & ANY & $A \boxplus B$ \\
        $\begin{bmatrix}1 & 0 & T \\ 0 & 0 & 0 \\ T & 0 & 1 \\\end{bmatrix}$ & Multiplication & MUL & $A \otimes B$ \\
        $\begin{bmatrix}1 & T & 0 \\ T & 0 & 1 \\ 0 & 1 & T \\\end{bmatrix}$ & Addition & SUM & $A \oplus B$ \\
    \end{tabular}
    \caption{Useful Two-Input Ternary Gates}
    \label{fig:two_value_ter_gates}
\end{figure}

All of the gates in Figure \ref{fig:two_value_bin_gates} and \ref{fig:two_value_ter_gates} have inverted equivalents that can be
created by negating the output will be denoted with an N proceeding the gate name. e.g. NAND, NOR, NANY.\\

\subsubsection{Assumptions}

Researchers have attempted to investigate this field using two distinct approaches. One is from a more theoretical perspective
where they investigate the potential gains of ternary logic. The other is from a more practical approach where 
Ternary test benches have been created in order to attempt to validate the theoretical performance gains.\\
\\
However, the underlying implementations of the basic ternary logic is underpinned by Complementary Metal Oxide 
Semiconductor (CMOS) along with it's PMOS and MNOS. Thus the fundamental reliance on the binary nature of the underlying 
hardware limits the performance gains that can be achieved.
For this project we will assume that comparable binary and ternary logic gates have the same propagation delay. This will allow us to
to directly compare the performance of the two systems. Implications of this assumption will be discussed in the conclusion.\\

\begin{equation}
\begin{aligned}
  AND &\equiv \mathbb{T}AND \\
  OR &\equiv \mathbb{T}OR \\
  NOT &\equiv \mathbb{T}NOT
\end{aligned}
\end{equation}

This will allow us to create a basic binary and ternary emulator whose performance will be directly comparable.

\subsection{Previous Ternary Implementations}

Setun was a sequential computer developed in 1958 by Sergei Sobolev and Nikolay Brusentov at Moscow State University. It was the most 
modern ternary computer using the balanced ternary numeral system consisting of 81 words of memory where each word was 
composed of 18 trits.  It has a one-address architecture with one index register and was able to handle both positive and 
negative floating point calculations.\\ 
% citep{weatherby2018ternary}.\\
% cite{brousentsov2002development}.\\
\\
The instruction set consisted of 24 instructions including performing mantissa normalisation for floating point calculations,
shift, combined multiplication and addition.
\\
In 1970, the Setun-70 was developed that built on idea of its predecessor. It had a short instruction set that was developed 
and implemented independently from the RISC architecture principles but did implement Dijkstra's ideas of of structured 
programming. Both of these computers where only used for research purposes and were never used commercially. The interval range 
for the mantissa value of a normalised number was changed to $\{0.5 < |m| < 1.5\}$ whereas it had been $\{0.1666... < |m| < 0.5\}$ 
in its predecessor.\\
% cite{brousentsov2002development}.\\
\\
During a thawing period of the Cold War, visits were allowed between scientists from the different nations resulting in 
TERNAC. This was an emulator written in FORTRAN by the State University of New York at Buffalo and was able to deliver 
on technical promise of ternary systems. As Setun when implementing the hardware was lacking a switch that could encode 
the two values in one setting and instead used two separate gates.\\
%citep{weatherby2018ternary}.\\
\\
Weatherby summarises this history with the following statement:
\begin{displayquote}
    "Ternary computing, in other words, has always been emulation rather than implementation—its history is in large part imaginary."
\end{displayquote}
And while his article is not a scientific paper and is written in a stylised and opinionated manner, it does raise a valid point.  
Fundamentally it may be impossible to implement a true ternary system in hardware due to fundamental laws around digital logic. 
However, we need not concern ourselves as we are only looking at the implications of ternary logic should it be possible 
to implement in hardware.\\
\\
Brousentov takes an opposing point of view in his paper where he argues that the Setun computer did realise the theoretical 
benefits despite the two separate gates:
% citep{brousentsov2002development}.\\
\begin{displayquote}
    "This experience convincingly confirms practical preferences of ternary digital technique."
\end{displayquote}
This also should be viewed with caution as the paper was written by Russian scientists in 2002 and while it gives a few details on the 
architecture of Setun it does not contain any quantitative measures. It does contain a lot uncited claims with a heavy use of adjectives of 
almost a persuasive nature.\\
\\
Ternary logic is used to encode 3 different pulse amplitudes on a single wire on the memory bus of NVIDIA 30 series GPUs and will be used in 
their next generation of GPUs and is currently in development by Micron.
% \citep{micron2024gddr7}.\\
PAM3 encoding is used to increase the single pin throughput 32Gb/s compared to its predecessor GDDR6 which had a throughput of 18Gb/s. This 
is achieved by encoding the sequential binary values into ternary values represented by 3 different voltage levels. This allows for 3 bits 
of information to be transmitted on a single wire within two clock cycles (1.5 bits per cycle).\\
% \citep{micron2024gddr7}.\\
% \citep{nam2024low}.\\

\subsection{Emulation of Circuits}

\newpage

% \bibliography{references}

\end{document}